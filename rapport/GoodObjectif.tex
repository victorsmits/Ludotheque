\subsection{Simplicité}
    \paragraph{}
        L'utilisation du design pattern singleton est un choix intelligent en ce qui concerne les performances du programme.
        En effet l'utilisation d'un singleton améliore l'occupation en mémoire et la vitesse d'exécution en ne générant qu'une seul instance de l'objet \textit{Manager}.
        
\vspace{\baselineskip}       
\subsection{Test unitaire}
    \paragraph{}
        L'intégrité de l'entièreté du projet est assurée par les tests unitaires. Ils nous permettent de vérifier que chaque modifications ou améliorations apportées n'amène pas le programme à un dysfonctionnement.

\vspace{\baselineskip}       
\subsection{Rôle des utilisateurs}
    \paragraph{}
        Chaque utilisateur a des droits précis quant aux actions qu'il peut réaliser dans le programme. La distinction entre un client et un manager est assurée par la présence de 2 classe ( \textit{Adherent , Manager} ) ayant chacune des actions différentes sur le programme. Toutefois, certaines actions restent communes aux deux types d'utilisateurs. Pour rendre cela possible, l'équipe a appliqué l'héritage sur ces 2 classes en créant une classe commune, la classe \textit{Person}. Ainsi nous pouvons partager certaines méthodes entre le manger et l'adhérent tout en évitant la duplication de code.