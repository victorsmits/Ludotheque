\paragraph{}
    Dans le cadre du cours \textit{Architecture logicielle}, il nous a été demandé, dans un premier temps d'établir un projet en \textit{Java}. L'important lors de la réalisation de ce projet était de mettre en place différents critères de qualité et de contrôler ceux-ci. Après 4 séances, les différents projets ont été redistribués à travers les groupes dans l'optique d'améliorer la qualité du programme et non sa fonctionnalité.

\paragraph{}
    Une fois le nouveau projet récupéré, il convient d'en faire une analyse approfondie ainsi qu'une analyse des différents critères de qualité mis en place. C'est sur base de ces critères que nous effectuerons notre travail de refactoring du code. Dans ce dossier se trouve les différentes analyses, problématiques, solutions et améliorations relatives à ce projet. Les objectifs atteints ou non y sont détaillés ainsi que les modifications apportées.

\paragraph{}
    L'ensemble du projet peut être trouver sur \textit{Github} \href{https://github.com/victorsmits/Ludotheque}{en cliquant ici} ou à l'adresse suivante: https://github.com/victorsmits/Ludotheque.

\paragraph{}
    L'intégration continue du projet a été réaliser grâce au système Jenkins et est retrouvable au \href{https://jenkins.ecam.be/login}{en cliquant ici} ou en vous rendant sur la platform jenkins de l'ecam dans le projet \textit{Ludotech}.