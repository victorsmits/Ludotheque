\paragraph{}
Pour clôturer ce travail, il convient de rappeler que l'analyse du projet existant nous a permis de déterminer la qualité du code. Ceci grâce aux différents critères de qualité qui ont été mis en place et décidés par la première équipe de développement. Certains de ces critères étaient respectés et appliqués correctement dans le code. Le couplage entre classes est en effet assez faible, ce qui est un indicateur qu'il ne sera pas trop difficile d'apporter des modifications au projet; il est relativement extensible, malgré quelques défauts. 
\\
Toutefois une partie des critères n'ont pas été respectés ou pas entièrement. Il convenait donc de modifier le code pour en augmenter la qualité et respecter les critères mis en place. La complexité du projet est également relativement basse ou moyenne; cependant la cohésion est globalement assez mauvaise, ce qui indique que des améliorations peuvent être apportées au projet pour augmenter sa maintenabilité. 

\paragraph{}
Enfin, il convient de dire qu'il n'y pas de "recette" quant à la qualité d'une application. Les critères qui détermine cette qualité doivent être défini au préalable par l'équipe de développement selon les besoins de l'application. Il ne faut pas se fier à un seul critère de qualité ni se jeter sur ceux que l'on connaît. Il est donc important d'effectuer une analyse approfondie des besoins que l'application doit remplir afin d'en déterminer les critères de qualité adéquats.